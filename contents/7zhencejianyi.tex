\section{政策建议}
  \label{sec:zhencejianyi}

  \subsection{建议一}
    我们根据\emph{模型评价与改进}中的灰色预测模型得到了影响因素的重要性排序。结果:
    \[
      \text{总人口}>\textrm{GDP}>\text{其他因素}
    \]
    可见\textbf{总人口}和 \textbf{GDP} 对碳排放量影响最大,其他影响因素之间相差无几。
    
    在其中,\textbf{总人口}一定程度上代表全部人口需要的生产物资、\textbf{GDP} 代表全年所有商品生产的价值总和。需要生产的产品越多,所消耗的能源也越多,因此碳排放量也就越大。
    
    在改革不能自己把自己革命了的前提下,我们提出:
    \begin{enumerate}
      \item 坚持计划生育的基本国策。
      \item 加快中国 $\mathrm{GDP}$ 由高速度增长向高质量发展的经济转型;去杠杆、降产能,稳中求进,积极转变。
      \item 转移城市过剩生产力,帮助农村发展,减少城乡差异;精准扶贫,建设全面小康。
    \end{enumerate}

  \subsubsection{建议二}
    另外,碳排放量在能源消耗一定的情况下,与能源结构、能源利用率有较大关联。
    因此我们提出:
    \begin{enumerate}
      \item 提高化石能源利用率,关闭一些高耗能、高污染、技术落后的小、中型发电厂,加快新技术的研发和推广。
      \item 积极推广绿色能源(风电、水电等)、积极研发新能源,减少对化石能源的依赖。
    \end{enumerate}