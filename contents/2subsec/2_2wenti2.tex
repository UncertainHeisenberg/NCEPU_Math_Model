\subsection{问题 2:碳排放预测模型}

  \subsubsection{碳排放影响因素分析}
    为建立模型,需要对碳排放的影响因素进行分析。根据有关文献,碳排放影响因素一般包括\cite{NCEPU2019}:
    \begin{enumerate}
      \item 人口因素;
      \item 城镇化率;
      \item 经济发展水平:人均 $\mathrm{GDP}$ 或者消除价格波动影响的国内生产总值;
      \item 人均碳排放量;
      \item 能源消费强度:能源消费量与 $\mathrm{GDP}$ 之比;
      \item 能源消费结构:各种能源所占比例,可以用煤炭比例来表示;
      \item 产业结构:三类产业占比,可以用第二产业占比表示;
      \item 国际贸易:出口额占 $\mathrm{GDP}$ 比重。
    \end{enumerate}
    我们选取,如表 \ref{tab:bianliangshuoming} 中变量,作为主要影响因素并确定其计算方法。

  \subsubsection{模型选择}
    由上述可知,影响中国碳排放量的因素繁多,此外碳排放量与影响因素之间不一定线性关系。

    $\mathrm{BP}$ 神经网络实现了一个从输入到输出的映射功能,而理论上一个有无限个隐层节点的神经网络具有实现任何复杂非线性映射的功能。这使得它特别适合于求解内部机制复杂的问题。\cite{liuzhongqi2010}

    此外,该模型还拥有自我学习能力,能通过学习大量、正确的实例,提取出规律,完善、改进原有模型。

    总之,由于 $\mathrm{BP}$ 神经网络模型具有较强的自组织、自适应与自学习能力,j非线性映射能力,泛化能力以及容错能力。我们选用 $\mathrm{BP}$ 神经网络预测模型对我国碳排放量进行预测。

    \begin{table}[htb]
      \centering
      \caption{模型变量说明}
      \begin{tabular*}{0.618\paperwidth}{@{\extracolsep{\fill}}ccccc}
        \toprule[1.5pt]
        &变量 && 计算方法 &\\
        \midrule[1pt]
        &碳排放量 && 各能源按碳排放系数求和(参见 \hyperref[eq:IPCC]{IPCC 方法}) &\\
        &城镇化率 && 城镇人口与总人口比值 &\\
        &$\mathrm{GDP}$ && 各产业生产总值之和 &\\
        &产业结构 && 第二产业与 $\mathrm{GDP}$ 比值 &\\
        &总人口数 && 城乡人口之和 &\\
        &经济发展水平 && $\mathrm{GDP}$ 与人口比值 &\\
        &国际贸易 && 出口额与 $\mathrm{GDP}$ 比值 &\\
        &人均碳排放量 && 碳排放量与总人口比值 &\\
        &能源消费强度 && 能源消费量与 $\mathrm{GDP}$ 比值 &\\
        \bottomrule[1.5pt]
      \end{tabular*}
      \label{tab:bianliangshuoming}
    \end{table}