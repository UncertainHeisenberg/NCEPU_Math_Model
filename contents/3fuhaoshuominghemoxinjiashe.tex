\section{符号说明和模型假设}

  \subsection{符号说明}
    \begin{table}[hb]
      \caption{符号说明}
      \label{tab:fuhaoshuoming}
      \centering
      \begin{tabular*}{0.8\textwidth}{@{\extracolsep{\fill}}ccccc}
        \toprule[1.5pt]
        &符号 && 说明 &\\
        \midrule[1pt]
        &$Y$ && 碳排放总量 &\\
        &$X1$ && 总人口数 &\\
        &$X2$ && $\mathrm{GDP}$ &\\
        &$X3$ && 产业结构 &\\
        &$X4$ && 城镇化率 &\\
        &$X5$ && 经济发展水平 &\\
        &$X6$ && 国际贸易 &\\
        &$X7$ && 人均碳排放量 &\\
        &$X8$ && 能源消费强度 &\\
        \bottomrule[1.5pt]
      \end{tabular*}
    \end{table}

  \subsection{模型假设}
    \begin{enumerate}
      \item 没有外在的、突发的影响或变化,如:能源革命,能源枯竭……即总体碳排放是以某种趋势变化的,总体能源结构稳定;
      \item 限定碳排放主要影响因素在总人口,GDP,产业结构,城镇化率,经济发展水平,国际贸易,人均碳排放量,能源消费强度之中;
      \item 不考虑给定的数据的资金时效性,及给定的 $\mathrm{GDP}$ 已消除价格波动影响。
      \item 使用的数据真实有效。
    \end{enumerate}