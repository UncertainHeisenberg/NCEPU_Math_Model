\section{问题重述}

  \subsection{问题背景}
    受温室效应的影响,全球气候不断恶化,这严重影响了自然生态环境和人类生活环境。其中二氧化碳等温室气体大规模排放被认为是引起温室效应的主要原因。

    当前,中国碳排放量已经超过美国和欧洲碳排放量之和,排放的温室气体,占全球温室气体排放总量的逾四分之一\cite{huanganwei2018}。中国作为全球最大的碳排放国家,面临着巨大的减排压力。

    从可持续发展的角度来看,探索碳排放增长的内在因素、开展碳减排策略的研究,对我国实施碳排放减排政策和低碳经济发展战略,以至减缓全球温室效应增长,具有重要的理论和实际意义。

    碳排放核算方法采用联合国提供的 IPCC 方法:
    \begin{equation*}
      C=\sum E_i \times F_i
      \label{eq:IPCC}
    \end{equation*}
    其中$C$ 为总碳排放量,$E_i$ 为能源 $i$ 消耗量,$F_i$ 为能源 $i$ 的碳排放系数。\\各系数取值参见表 \ref{tab:tanpaifanxishu} 所示。
    \begin{table}[hb]
      \centering
      \caption{各化石能源碳排放系数}
      \begin{tabular}{|l|l|l|l|}
        \hline
        能源 & 煤炭 & 石油 & 天然气 \\
        \hline
        碳排放系数 & $0.7476$ & $0.5825$ & $0.4435$ \\
        \hline
      \end{tabular}
      \label{tab:tanpaifanxishu}
    \end{table}

    \hyperref[ssec:fujian1]{附件 1} 中列出了中国近些年来的一些经济数据。

  \subsection{问题提出}
    \begin{description}
      \item[问题 1. ] 分析我国能源消费结构。
      \item[问题 2. ] 确定碳排放量影响因素,并建立碳排放预测模型。
      \item[问题 3. ] 根据模型,提出节能减排政策建议。 
    \end{description}
