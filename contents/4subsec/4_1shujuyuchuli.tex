\subsection{数据预处理}

  \subsubsection{初始数据计算}

    根据\emph{问题分析}中问题二的\hyperref[tab:bianliangshuoming]{\emph{变量计算方法}},我们通过 $\mathrm{Excel}$ 处理\hyperref[ssec:fujian1]{附件 1} 中数据,得到、整理出了 $2000\sim 2016$ 年,碳排放量及其八个变量的具体值。(数据如表 \ref{tab:jutizhi} 所示,具体文件参见 \hyperref[ssec:fujian2]{\emph{附件 2}})
    % Table generated by Excel2LaTeX from sheet 'Sheet1'
    \begin{table}[hb]
      \centering
      \caption{2000--2016 年各变量具体值}
      \resizebox{\textwidth}{!}{
        \begin{tabular}{rrrrrrrrrr}
        \toprule[1.5pt]
        \multicolumn{1}{l}{年份} &
        \multicolumn{1}{l}{碳排放量} & \multicolumn{1}{l}{城镇化率} & \multicolumn{1}{l}{$GDP$} & \multicolumn{1}{l}{产业结构} & \multicolumn{1}{l}{总人口数} & \multicolumn{1}{l}{经济发展水平} & \multicolumn{1}{l}{国际贸易} & \multicolumn{1}{l}{人均碳排放量} & \multicolumn{1}{l}{能源消费强度} \\
        \midrule[1pt]
        2000 & 95528.51053 & 0.362197518 & 100280.1 & 0.455372502 & 126743 & 0.791208193 & 0.205763656 & 0.753718237 & 1.465535036 \\
        2001 & 99939.25859 & 0.376597428 & 110863.2 & 0.447945757 & 127627 & 0.868650051 & 0.198659249 & 0.783057336 & 1.403053493 \\
        2002 & 109314.6759 & 0.390897838 & 121717.4 & 0.444517382 & 128453 & 0.9475637 & 0.221398091 & 0.851009131 & 1.393202615 \\
        2003 & 128517.4696 & 0.405302298 & 137422 & 0.456239903 & 129227 & 1.06341554 & 0.264061795 & 0.994509426 & 1.43414446 \\
        2004 & 149897.5482 & 0.417600086 & 161840.1 & 0.459014175 & 129988 & 1.245038773 & 0.30340441 & 1.153164509 & 1.422892102 \\
        2005 & 171351.2686 & 0.429899966 & 187318.9 & 0.470237654 & 130756 & 1.432583591 & 0.334445697 & 1.310465819 & 1.395315689 \\
        2006 & 187685.8448 & 0.443430102 & 219438.5 & 0.475585642 & 131448 & 1.669393981 & 0.353617073 & 1.427833401 & 1.305454603 \\
        2007 & 203788.9583 & 0.458892446 & 270232.3 & 0.468610155 & 132129 & 2.04521566 & 0.346233962 & 1.542348449 & 1.152497314 \\
        2008 & 207400.2101 & 0.469895032 & 319515.6 & 0.469324815 & 132802 & 2.40595473 & 0.314209827 & 1.56172505 & 1.003428315 \\
        2009 & 217249.6957 & 0.48341701 & 349081.4 & 0.458837681 & 133450 & 2.615821656 & 0.234987295 & 1.627948263 & 0.962887166 \\
        2010 & 229528.7214 & 0.499496611 & 413030.4 & 0.463960522 & 134091 & 3.080224624 & 0.25911614 & 1.711738456 & 0.873175437 \\
        2011 & 248898.1417 & 0.512702713 & 489300.5 & 0.464006883 & 134735 & 3.631576799 & 0.251870987 & 1.847316152 & 0.791012885 \\
        2012 & 254319.7118 & 0.525700866 & 540367.5 & 0.452735037 & 135404 & 3.990779445 & 0.239390785 & 1.878228943 & 0.744193535 \\
        2013 & 261402.5332 & 0.537296431 & 595244.5 & 0.440081513 & 136072 & 4.374481892 & 0.230378273 & 1.921060418 & 0.700406304 \\
        2014 & 262747.8929 & 0.547703645 & 643973.9 & 0.431029581 & 136782 & 4.708031027 & 0.22343095 & 1.920924485 & 0.661216239 \\
        2015 & 261805.7824 & 0.560998676 & 689052.1 & 0.409316364 & 137462 & 5.012673321 & 0.20487101 & 1.904568407 & 0.623907829 \\
        2016 & 260834.8387 & 0.573496973 & 744127.2 & 0.398098605 & 138271 & 5.381657759 & 0.186015634 & 1.886403068 & 0.585678094 \\
        \bottomrule[1.5pt]
        \end{tabular}}%
      \label{tab:jutizhi}%
    \end{table}%

  \subsubsection{数据分类}

    之后我们把数据分类,分别作为 $\mathrm{BP}$ 神经网络的训练数据集、检验数据集和最后的预测数据集。详细如表 \ref{tab:shujufenlei}。

    \begin{table}[thb]
      \caption{数据分类}
      \label{tab:shujufenlei}
      \centering
      \begin{tabular*}{0.8\textwidth}{@{\extracolsep{\fill}}ccccc}
        \toprule[1.5pt]
        &数据集 && 年份 &\\
        \midrule[1pt]
        &训练数据集 && 2000--2013(包括 2013) &\\
        &检验数据集 && 2014--2015(包括 2015) &\\
        &预测数据集 && 2016 &\\
        \bottomrule[1.5pt]
      \end{tabular*}
    \end{table}

  \subsubsection{数据归一化处理}

    此外,为去除数据量纲,方便数据处理,并使 $\mathrm{BP}$ 神经网络收敛加快,我们将数据集进行了归一化处理。