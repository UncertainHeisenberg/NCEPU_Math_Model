\subsection{问题 2:碳排放预测模型}

  \subsubsection{碳排放影响因素分析}
    为建立模型,需要对碳排放的影响因素进行分析。根据有关文献,碳排放影响因素一般包括:
    \begin{enumerate}
      \item 人口因素;
      \item 城镇化率;
      \item 经济发展水平:人均 $\mathrm{GDP}$ 或者消除价格波动影响的国内生产总值;
      \item 人均碳排放量;
      \item 能源消费强度:能源消费量与 $\mathrm{GDP}$ 之比;
      \item 能源消费结构:各种能源所占比例,可以用煤炭比例来表示;
      \item 产业结构:三类产业占比,可以用第二产业占比表示;
      \item 国际贸易:出口额占 $\mathrm{GDP}$ 比重。
    \end{enumerate}
    我们选取,如表 \ref{tab:bianliangshuoming} 中变量,作为主要影响因素并确定其计算方法。

  \subsubsection{模型选择}
    由上述可知,影响中国碳排放量的因素繁多,此外碳排放量与影响因素之间是非线性关系。

    BP神经网络是一种按误差反向传播(简称误差反传)训练的多层前馈网络,其算法称为BP算法,它的基本思想是梯度下降法,利用梯度搜索技术,以期使网络的实际输出值和期望输出值的误差均方差为最小。

    BP神经网络模型具有较强的自组织、自适应与自学习能力,非线性映射能力,泛化能力以及容错能力,因此我们选用BP神经网络预测模型对我国碳排放量进行预测。

    \begin{table}[htb]
      \caption{模型变量说明}
      \label{tab:bianliangshuoming}
      \centering
      \begin{tabular*}{\textwidth}{@{\extracolsep{\fill}}ccccc}
        \toprule[1.5pt]
        &变量 && 计算方法 &\\
        \midrule[1pt]
        &碳排放量 && 各能源按碳排放系数求和(参见问题重述中:\hyperref[eq:IPCC]{IPCC 方法}) &\\
        &总人口数 && 城乡人口之和 &\\
        &$\mathrm{GDP}$ && 各产业生产总值之和 &\\
        &产业结构 && 第二产业与 $\mathrm{GDP}$ 比值 &\\
        &城镇化率 && 城镇人口与总人口比值 &\\
        &经济发展水平 && $\mathrm{GDP}$ 与人口比值 &\\
        &国际贸易 && 出口额与 $\mathrm{GDP}$ 比值 &\\
        &人均碳排放量 && 碳排放量与总人口比值 &\\
        &能源消费强度 && 能源消费量与 $\mathrm{GDP}$ 比值 &\\
        \bottomrule[1.5pt]
      \end{tabular*}
    \end{table}