\section{模型改进}
  由于 $\mathrm{BP}$ 神经网络像一个 ”黑匣子“ 一样,难以清晰明了地反映出各个因素对碳排放量的影响程度。

  为了确定不同因素对中国碳排放量影响程度的大小,并便于问题三的解决。我们建立了灰色关联度分析模型来对数据进行量化比较分析。
  
  具体通过求解出各个因素对中国碳排放量的关联度,并按关联度大小排序,从而得到得到各影响因素的重要性排序。

  \subsection{灰色关联分析模型简介}
    \cite{guozhonghua2008}灰色关联度分析法是一种多因素统计分析方法。它以各因素的样本数据为依据,用灰色关联度来描述各因素间关系的强弱、大小和次序。
    
    若样本数据反映出的两因素变化的态势(方向、大小和速度等)基本一致,则它们之间的关联度较大;反之,关联度较小。
    
    灰色关联度分析对于一个系统发展变化态势提供了量化的度量,这非常适合动态历程分析。

  \subsection{灰色关联分析模型的建立}

    下列过程基本参照\textbf{灰色关联分析及 $\mathrm{MATLAB}$ 程序实现}\cite{de2013} 的结构。

    \subsubsection{确定分析序列}

      在对本问题定性分析的基础上,我们分别以八个因素在 2000--2016 年的\textbf{已归一化}的数据作为八个比较序列。
      \begin{align*}
        \intertext{设 $8$ 个比较序列形成的矩阵为:}
        (X'_1,X'_2,\ldots,X'_8) &= 
        \begin{bmatrix}
          X_1(2000) & X_2(2000) & \cdots & X_8(2000) \\
          X_1(2001) & X_2(2001) & \cdots & X_8(2001) \\
          \vdots & \vdots & \vdots & \vdots \\
          X_1(2016) & X_2(2016) & \cdots & X_8(2016) \\
        \end{bmatrix}
        \intertext{则单个比较序列为:}
        X'_i &= \bigl(X_i(2000),X_i(2001),\dots,X_i(2016)\bigr)^T
        \intertext{当中 $X_i(\text{年份})$ 表示该年份的 $X_i$ 的归一化后的值。}
      \end{align*}

      以中国碳排放量在 2000--2016 年\textbf{已归一化}的数据作为参考数列,参考数量作为理想的比较标准。
      \[
        Y' = [Y(2000),Y(2001),\ldots,Y(2016)]^T
      \]
      同理,$Y(\text{年份})$ 表示该年份的 $Y$ 值(碳排放量)。

    \subsubsection{计算关联度}
      \begin{enumerate}
        \item 逐个计算每个比较序列对应元素与参考序列对应元素的\textbf{绝对差值}。即:$\Delta_i(k)=|Y(k)-X_i(k)|\qquad (k=2000,\dots,2016\quad i=1,\dots,8)$
        \item 对于任意的 $i,k$ 在其定义范围内的取值,确定绝对差值的最大值和最小值。即:$\Delta_\text{max}$ 和 $\Delta_\text{min}$ 。
        \item 由下式,分别计算每个比较序列对应元素与参考序列对应元素的\textbf{关联系数}。即:
          \[
            \zeta_i(k)=\frac{\Delta_\text{min}+\rho\cdot\Delta_\text{max}}{\Delta_i(k)+\rho\cdot\Delta_\text{max}}
          \]
          其中 $\rho$ 为分辨系数($0<\rho<1$)。\\
          若 $\rho$ 越小,则关联系数间的差异越大,区分能力也就越强。通常取 $\rho$ 为 0.5。
        \item 对各比较序列再分别求其 17 个关联系数的均值,并称该值为\textbf{关联度}\cite{de2013}。即:
          \[
            r_i=\frac{1}{17}\sum_{k=2000}^{2016} \zeta_i(k)
          \]
      \end{enumerate}

    \subsubsection{模型求解}
      利用 $\mathrm{MATLAB}$ 编程实现上述语句。通过带入八个影响因素以及碳排放总量在 2000--2016 年的相关数据,得到其关联度及其排名。如表 \ref{tab:guanlianxu0.5} 所示。
      \begin{table}[htb]
        \centering
        \caption{各影响因素关联度所求解及排序}
        \begin{tabular*}{0.618\paperwidth}{@{\extracolsep{\fill}}ccccccc}
          \toprule[1.5pt]
          &影响因素 && 关联度 && 排名 &\\
          \midrule[1pt]
          &总人口数 && 0.6746 && 1 &\\
          &$\mathrm{GDP}$ && 0.6066 && 2 &\\
          &产业结构 && 0.3949 && 3 &\\
          &城镇化率 && 0.3949 && 3 &\\
          &经济发展水平 && 0.3949 && 3 &\\
          &国际贸易 && 0.3949 && 3 &\\
          &人均碳排放量 && 0.3949 && 3 &\\
          &能源消费强度 && 0.3949 && 3 &\\
          \bottomrule[1.5pt]
        \end{tabular*}
        \label{tab:guanlianxu0.5}
      \end{table}

      由表 \ref{tab:guanlianxu0.5},我们得到影响因素重要排序。即:
      \[
        \text{总人口}>\textrm{GDP}>\text{其他因素}
      \]