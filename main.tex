\documentclass[hyperref, UTF8, cs4size, titlepage]{ctexart}
\usepackage{bookmark}

% 图表标题
\usepackage{caption}

% 图片
\usepackage{graphicx}

% 附件添加
\usepackage{attachfile}

% 数学
\usepackage{amsmath}

% 超文本链接设置
\hypersetup{colorlinks=true, linkcolor=blue, filecolor=magenta, urlcolor=cyan, pdfpagemode=FullScreen}

% 自定义命令
\newcommand{\keywords}[1]{\textbf{\textit{关键字---}} #1}

% 版面制作
\usepackage[a4paper, left=2.5cm, right=2.5cm, top=2.5cm, bottom=2.5cm]{geometry}
% 页面(页尾设置)
\pagestyle{plain}

% 行距设置
%\usepackage{setspace}
%\singlespacing

% 目录拓展
\usepackage[nottoc]{tocbibind}

% 引用设置
\bibliographystyle{plain}

% 索引设置
\usepackage{makeidx}
\makeindex

% 标题设置
\usepackage[raggedright]{titlesec}
\titleformat*{\section}{\zihao{4}\upshape\bfseries\centering}

\begin{document}

  % 添加封面
  \begin{titlepage}
  \begin{center}
    \vspace*{\fill}
    {\zihao{3}\bfseries 选题:$\mathrm{A}$}
    \vspace*{\fill}

    {\zihao{3}\bfseries 基于 $\mathrm{BP}$ 神经网络的碳排放预测及其政策建议}
    \vspace*{\stretch{2}}

    \zihao{4}\itshape
    \begin{tabular}{ccc}
      建模:& 成宇琛 & 电气 1813 班 \\
      编程:& 孙瑞 & 计算机 1806 班 \\
      论文:& 邓思彤 & 电气 1813 班 \\
    \end{tabular}
    \vspace*{\stretch{3}}

    \today
  \end{center}
  \vspace*{\stretch{1}}
\end{titlepage}

  % 添加摘要
  \phantomsection
  \begin{abstract}
  \thispagestyle{plain}
  % TODO: 摘要
  称为温室效应。\index{温室效应}

  \vspace{\stretch{3}}
  {\keywords{温室效应,碳排放}}
  \vspace{\fill}
\end{abstract}
  \addcontentsline{toc}{section}{摘要}
  \clearpage

  % 添加目录
  %\phantomsection
  \tableofcontents
  \thispagestyle{empty}
  \clearpage

  % 内容

  \section{问题重述}

  \subsection{问题背景}
    受温室效应的影响,全球气候不断恶化,这严重影响了自然生态环境和人类生活环境。其中二氧化碳等温室气体大规模排放被认为是引起温室效应的主要原因。

    当前,中国碳排放量已经超过美国和欧洲碳排放量之和,排放的温室气体,占全球温室气体排放总量的逾四分之一\cite{}。中国作为全球最大的碳排放国家,面临着巨大的减排压力。

    从可持续发展的角度来看,探索碳排放增长的内在因素、开展碳减排策略的研究,对我国实施碳排放减排政策和低碳经济发展战略,以至减缓全球温室效应增长,具有重要的理论和实际意义。

    碳排放核算方法采用联合国提供的 IPCC 方法:
     \[C=\sum E_i \times F_i\]
    其中$C$ 为总碳排放量,$E_i$ 为能源 $i$ 消耗量,$F_i$ 为能源 $i$ 的碳排放系数。\\各系数取值参见表 \ref{tab:tanpaifanxishu} 所示。
    \begin{table}[hb]
      \caption{各化石能源碳排放系数}
      \label{tab:tanpaifanxishu}
      \centering
      \begin{tabular}{|l|l|l|l|}
        \hline
        能源 & 煤炭 & 石油 & 天然气 \\
        \hline
        碳排放系数 & $0.7476$ & $0.5825$ & $0.4435$ \\
        \hline
      \end{tabular}
    \end{table}

    \hyperref[sec:fujian1]{附件 1} 中列出了中国近些年来的一些经济数据。

  \subsection{问题提出}
    \begin{description}
      \item[问题 1. ] 分析我国能源消费结构。
      \item[问题 2. ] 确定碳排放量影响因素,并建立碳排放预测模型。
      \item[问题 3. ] 根据模型,提出节能减排政策建议。 
    \end{description}

  \clearpage

  \section{问题分析}

  \subsection{问题 1:我国能源结构分析}

    我们将煤炭、石油、天然气和一次电力及其他等能源占能源消费总量的比重,按时间序列作出折线图(见图 \ref{fig:nenyuanxiaofeibizhong})。
    \begin{figure}[hb]
      \centering
      \includegraphics[scale=0.6]{figures/fig1.png}
      \captionsetup{format=hang}
      \caption[煤炭、石油、天然气和一次电力及其他等能源占能源消费总量的比重]{蓝色曲线为煤炭占总能源的比例,红色为石油,绿色为天然气,青色为一次电力及其他能源。}
      \label{fig:nenyuanxiaofeibizhong}
    \end{figure}

    由图 \ref{fig:nenyuanxiaofeibizhong} ,沿 $y$ 轴可以分析出以下几点:
    \begin{enumerate}
      \item 煤炭消费占我国能源消费的首要地位,其比例始终居于 $60\%$ 之上。
      \item 石油消费为最主要的辅助能源,其比例一直在 $20\%$ 附近。
      \item 一次电力能源与天然气在总能源中的比例相对较低,大体上低于 $10\%$ 。
    \end{enumerate}

    再沿时间轴可以分析出:
    \begin{enumerate}
      \item 煤炭消费在波动中整体呈现缓步下降趋势,但仍然占据我国能源消费的首要地位。
      \item 石油消费在参考年限中呈稳定趋势,略有波动和下滑。但与一次电力及其他的距离不断拉近。
      \item 一次电力及其他能源和天然气能源消费是呈逐年上升趋势,且天然气的上升速度随年份增加而增加。但所占比例仍然不足。
    \end{enumerate}

    \subsubsection{总结}
    我国的能源消费结构始终为以煤炭消费为主导地位,并以石油消费为主要辅助消费,但也存在天然气、一次电力及其他能源等能源占据市场。能源消费市场呈现 “一超多强” 局面。
    
    天然气和一次电力这样的清洁能源与石油、煤炭消费所占的比重不断拉近,但仍未占据可观消费比例。这说明我国的能源消费结构在不断优化的过程中还有很大的发展空间。

  \subsection{问题 2:碳排放预测模型}

    \subsubsection{碳排放影响因素分析}
      为建立模型,需要对碳排放的影响因素进行分析。根据有关文献,碳排放影响因素一般包括:
      \begin{enumerate}
        \item 人口因素;
        \item 城镇化率;
        \item 经济发展水平:人均 $GDP$ 或者消除价格波动影响的国内生产总值;
        \item 人均碳排放量;
        \item 能源消费强度:能源消费量与 $GDP$ 之比;
        \item 能源消费结构:各种能源所占比例,可以用煤炭比例来表示;
        \item 产业结构:三类产业占比,可以用第二产业占比表示;
        \item 国际贸易:出口额占 $GDP$ 比重。
      \end{enumerate}
      我们选取% TODO 因素
      进行分析。
    
    \subsubsection{模型建立}

  \subsection{问题 3:政策建议}
    参见 \hyperref[sec:zhencejianyi]{7 政策建议}。
  \clearpage

  \input{contents/3muxinjiashe}

  \section{符号说明}
  \begin{table}[hb]
    \caption{符号说明}
    \label{tab:fuhaoshuoming}
    \centering
    \begin{tabular}{cc}
      \toprule[1.5pt]
      符号 & 说明 \\
      \midrule[1pt]
      $Y$ & 碳排放总量 \\
      $X1$ & 总人口数 \\
      $X2$ & $GDP$ \\
      $X3$ & 产业结构 \\
      $X4$ & 城镇化率 \\
      $X5$ & 经济发展水平 \\
      $X6$ & 国际贸易 \\
      $X7$ & 人均碳排放量 \\
      $X8$ & 能源消费强度 \\
      \bottomrule[1.5pt]
    \end{tabular}
  \end{table}

  \section{模型建立及求解}
  \clearpage

  \section{模型评价与改进}

  \section{政策建议}
  \label{sec:zhencejianyi}

  \subsection{建议一}
    我们根据\emph{模型评价与改进}中的灰色预测模型得到了影响因素的重要性排序。结果:
    \[
      \text{总人口}>\textrm{GDP}>\text{其他因素}
    \]
    可见\textbf{总人口}和 \textbf{GDP} 对碳排放量影响最大,其他影响因素之间相差无几。
    
    在其中,\textbf{总人口}一定程度上代表全部人口需要的生产物资、\textbf{GDP} 代表全年所有商品生产的价值总和。需要生产的产品越多,所消耗的能源也越多,因此碳排放量也就越大。
    
    在改革不能自己把自己革命了的前提下,我们提出:
    \begin{enumerate}
      \item 坚持计划生育的基本国策。
      \item 加快中国 $\mathrm{GDP}$ 由高速度增长向高质量发展的经济转型;去杠杆、降产能,稳中求进,积极转变。
      \item 转移城市过剩生产力,帮助农村发展,减少城乡差异;精准扶贫,建设全面小康。
    \end{enumerate}

  \subsubsection{建议二}
    另外,碳排放量在能源消耗一定的情况下,与能源结构、能源利用率有较大关联。
    因此我们提出:
    \begin{enumerate}
      \item 提高化石能源利用率,关闭一些高耗能、高污染、技术落后的小、中型发电厂,加快新技术的研发和推广。
      \item 积极推广绿色能源(风电、水电等)、积极研发新能源,减少对化石能源的依赖。
    \end{enumerate}
  \clearpage

  % 引用文献
  \phantomsection
  \bibliography{ref/math_model.bib}
  \clearpage

  % 索引
  \phantomsection
  \printindex

  % 附录
  \appendix

  \section{附件 1}
    中国近些年来的一些经济数据\footnote{来源:中国统计年鉴及中国能源统计年鉴}。
    \attachfile{appendix/fujian1.xlsx}
    \label{sec:fujian1}

  \section{附录 2}

\end{document}