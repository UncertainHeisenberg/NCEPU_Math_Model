\subsection{$\mathrm{BP}$ 神经网络函数、参数设定}

  \subsubsection{具体设定及初始化}
    参见表 \ref{tab:shedingchushihua} 所示:
    \begin{table}[thb]
      \centering
      \caption{函数设定及参数初始化}
      \begin{tabular*}{0.618\paperwidth}{@{\extracolsep{\fill}}ccccc}
        \toprule[1.5pt]
        &函数 && 设定 &\\
        \midrule[1pt]
        &隐层激励函数 && $\mathrm{tansig}$ &\\
        &输出层激励函数 && $\mathrm{logsig}$ &\\
        &网络训练函数 && $\mathrm{traingdx}$ &\\
        &网络性能函数 && $\mathrm{mes}$ &\\
        \midrule[1pt]
        \midrule[1pt]
        &参数 && 初始化 &\\
        \midrule[1pt]
        &期望误差最小值 $\mathrm{err-goal}$ && 0.0000001 &\\
        &最大循环次数 $\mathrm{max-epoch}$ && 5000 &\\
        &修正权值的学习速率 $\mathrm{lr}$ && 0.01 &\\
        \bottomrule[1.5pt]
      \end{tabular*}
      \label{tab:shedingchushihua}
    \end{table}

  \subsubsection{选取激励函数}

    \cite{Sallybin2013}$\mathrm{BP}$ 神经网络通常采用 $\mathrm{Sigmoid}$ 可微函数和线性函数作为网络的激励函数。

    我们选择 $\mathrm{S}$ 型正切函数 $\mathrm{tansig}$ 作为隐层神经元的激励函数。

    而由于网络的输出归一到 $\left[ -1, 1 \right]$ 范围内, 因此预测模型选取 $\mathrm{S}$ 型对数函数 $\mathrm{logsig}$ 作为输出层神经元的激励函数

  \subsubsection{选取训练函数和性能函数}

    $\mathrm{traingdx}$ 该函数运用梯度下降法来训练函数,而且在训练过程中,其学习速率是可变的,训练速度较快,不易陷入局部最小情况。因此我们最终选取 $\mathrm{traingdx}$。

    性能函数我们选取 $\mathrm{BP}$ 神经网络通常采用的 $\mathrm{mex}$ 函数。

  \subsubsection{选取学习速率}

    学习速率对 $\mathrm{BP}$ 神经网络具有重要影响作用。

    学习速率太小,网络学习缓慢,需要增加训练次数;学习速率太快,容易导致网络不收敛,影响训练的精度。

    我们最终选取学习速率为 $0.01$,最大循环次数为 $5000$ 次。