\subsection{$\mathrm{BP}$ 神经网络结构构建}

  \subsubsection{结构展示}
  最终网络结构示意图如图 \ref{tab:shenjinwanluo} \cite{Fauske2006}所示:
  \begin{figure}[hb]
    \caption{神经网络结构图}
    \label{tab:shenjinwanluo}
    \centering
    \def\layersep{2.5cm}
    \def\Layersep{5.0cm}
    \begin{tikzpicture}[shorten >=1pt,->,draw=black!50, node distance=\layersep]
        \tikzstyle{every pin edge}=[<-,shorten <=1pt]
        \tikzstyle{neuron}=[circle,fill=black!25,minimum size=17pt,inner sep=0pt]
        \tikzstyle{input neuron}=[neuron, fill=green!50];
        \tikzstyle{output neuron}=[neuron, fill=red!50];
        \tikzstyle{hidden neuron}=[neuron, fill=blue!50];
        \tikzstyle{annot} = [text width=4em, text centered]

        % Draw the input layer nodes
        \foreach \name / \y in {1,...,8}
        % This is the same as writing \foreach \name / \y in {1/1,2/2,3/3,4/4}
            \node[input neuron, pin=left:输入 $X_\y$] (I-\name) at (0,-\y) {};

        % Draw the hidden layer nodes
        \foreach \name / \y in {2,...,7}
            %\path[yshift=0.5cm]
                \node[hidden neuron] (H-\name) at (\layersep,-\y cm) {};

        % Draw the output layer node
        \node[output neuron, pin={[pin edge={->}]right:输出 $Y$}] (O) at (\Layersep,-4.5 cm) {};

        % Connect every node in the input layer with every node in the
        % hidden layer.
        \foreach \source in {1,...,8}
            \foreach \dest in {2,...,7}
                \path (I-\source) edge (H-\dest);

        % Connect every node in the hidden layer with the output layer
        \foreach \source in {2,...,7}
            \path (H-\source) edge (O);

        % Annotate the layers
        \node[annot,above of=I-1, node distance=1cm] (il) {输入层};
        \node[annot,right of=il] (hl) {隐藏层};
        \node[annot,right of=hl] {输出层};
    \end{tikzpicture}
\end{figure}
  \clearpage

  \subsubsection{设计网络输入层和输出层}
    首先设计网络输入、输出层。具体如表 \ref{tab:shurushuchu} 所示:
    \begin{table}[htb]
      \centering
      \caption{$\mathrm{BP}$神经网络输入、输出层}
      \begin{tabular*}{0.618\paperwidth}{@{\extracolsep{\fill}}ccccc}
        \toprule[1.5pt]
        &输入层 && 输出层 &\\
        \midrule[1pt]
        &$\mathrm{X_1}$\quad $\mathrm{X_2}$\quad $\mathrm{X_3}$\quad $\mathrm{X_4}$\quad $\mathrm{X_5}$\quad $\mathrm{X_6}$\quad $\mathrm{X_7}$\quad $\mathrm{X_8}$ && $\mathrm{Y}$ &\\
        \bottomrule[1.5pt]
      \end{tabular*}
      \label{tab:shurushuchu}
    \end{table}

  \subsubsection{选取隐含层节点数}

    隐含层节点数的选取是决定神经网络训练精度的关键,过多过少都会有很大影响。

    隐含层节点过多,能有效减少系统误差,但也会导致诸如:网络训练时间延长、训练容易陷入局部极小点的问题。从而降低网络的容错性和范化能力;而隐含层节点数过少,则又可能造成网络性能差或者网络根本不能被训练的问题\cite{zhangfaming2016}。

    因此隐含层节点数的选取,我们参考 $\mathrm{Kolmogorov}$ 定理,并采取下列经验公式\cite{Fauske2006}:
    \[
      J = \sqrt{m+n}+a \text{\qquad 其中 $m$ 为输入节点数,$n$ 为输出节点数,$a$ 的取值范围为 1--10}
    \]
    最终确定隐层节点数范围为 4--13。

    我们通过试凑法确定最佳隐层节点数。\\
    分别计算 $4,5,\dots,13$ 作为隐层节点数的最终 $\mathrm{MES}$ 值。其结果如表 \ref{tab:yincenjiedian}所示。
    \begin{table}[htb]
      \centering
      \caption{各隐层节点数对应 $\mathrm{MES}$ 值表}
      \begin{tabular*}{0.618\paperwidth}{@{\extracolsep{\fill}}ccccc}
        \toprule[1.5pt]
        &隐层节点数 && $\mathrm{MES}$ &\\
        \midrule[1pt]
        &4 && &\\
        &5 && &\\
        &6 && &\\
        &7 && &\\
        &8 && &\\
        &9 && &\\
        &10 && &\\
        &11 && &\\
        &12 && &\\
        &13 && &\\
        \bottomrule[1.5pt]
      \end{tabular*}
      \label{tab:yincenjiedian}
    \end{table}

    因此我们最终选取 6 为隐含层节点数。