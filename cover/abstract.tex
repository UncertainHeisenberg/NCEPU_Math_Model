\begin{abstract}
  \thispagestyle{plain}

  碳排放问题在我国已引起广泛的关注。为对我国实施碳排放减排政策和低碳经济发展战略提供决策依据和有效建议。现对外来几年的碳排放进行预测,并以此提供有效建议。

  \textbf{针对问题一:}
  
  本文做出了了煤炭、石油、天然气和一次电力及其他能源占我国能源消费整体的比重随年份变化的折线图。

  以此为依据展开对我国能源消费结构的分析。分析包括了沿y轴(能源所占比例)和沿x轴(年份变化)的大小与趋势的分析。

  最终得出了:我国能源结构以化石能源(煤炭尤其)为其始终主导的现状;和总体能源结构在不断优化,但仍有相当大发展空间的未来趋势。

  \textbf{针对问题二:}
  
  本文选取、整理了总人口数、$\mathrm{GDP}$、产业结构、城镇化率、经济发展水平、国际贸易、人均碳排放量、能源消费强度,共八个因素作为主要影响因素,并以此建立了\textbf{ $\mathrm{BP}$ 神经网络模型}。

  对于模型建立,本文首先对样本数据进行计算、分类和归一化处理。再确定模型结构(包括输出层、中间隐层和输入层),以及具体函数的选取和参数初始化。

  之后借助于 $\mathrm{MATLAB}$ 软件中的神经网络计算功能,对模型进行了合理训练和数据拟合。最终得到对应年份的碳排放量的模拟值和预测值。

  \textbf{针对问题三:}
  
  由于 $\mathrm{BP}$ 神经网络不能清晰反映出各因素对碳排放量影响程度,本文在\emph{模型评价与改进}中引入了\textbf{灰色关联分析模型},来评判各因素的影响程度。
  
  具体通过代入数据到关联系数公式中,得出影响因素数列对参考数列的关联度,以此对各影响因素进行了关联性排序。

  通过关联度大小排序也就得到了影响因素的重要性排序。其结果是:$\text{总人口数}>\mathrm{GDP}>\text{其他}$。其中,总人口数和$\mathrm{GDP}$ 关联度徘徊在 0.6 上下,其他值集中在 0.39 上下。

  最终我们根据影响因素的重要性,提出:
  \begin{enumerate}
    \item 加快中国 $\mathrm{GDP}$ 由高速度增长向高质量发展的经济转型。
    \item 坚持计划生育的基本国策。
    \item 转移城市过剩生产力,帮助农村发展,减少城乡差异。
    \item 寻求开发绿色能源、新能源,减少对化石能源的依赖。
  \end{enumerate}

  \vspace{\stretch{3}}
  {\keywords{碳排放,能源结构,$\mathrm{BP}$ 神经网络,灰色关联模型,关联度分析法}}
  \vspace{\fill}
\end{abstract}